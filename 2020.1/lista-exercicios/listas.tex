\documentclass[12pt,a4paper,oneside]{article}

\usepackage[utf8]{inputenc}
\usepackage[portuguese]{babel}
%\usepackage[T1]{fontenc}
\usepackage{amsmath}
\usepackage{amsfonts}
\usepackage{amssymb}


\usepackage{xcolor}
% Definindo novas cores
\definecolor{verde}{rgb}{0.25,0.5,0.35}
\definecolor{jpurple}{rgb}{0.5,0,0.35}

\author{\\Universidade Federal de Jataí (UFJ)\\Bacharelado em Ciência da Computação \\Lógica para Ciência da Computação - 2020.1 \\Prof. Esdras Lins Bispo Jr.}
\date{}

\title{
	\sc \huge Listas de Exercícios
	\\{\tt Versão 1.5}
}

\begin{document}

\maketitle

\section{Livro de Referência}
	\begin{itemize}
		\item  {\bf \color{blue} [L1]} ALENCAR FILHO, E. {\bf Iniciação à Lógica Matemática}, 1a Edição, Editora Nobel, 2002. {\color{blue} \bf Código Bib.: [510.6 ALE/ini]}.
	\end{itemize}
	
\section{Listas de Exercícios}

\begin{enumerate}

	\subsection{Mini-Teste 1}
	\item[] {\bf Lista de Exercícios 01:} 1.1 (a-h), 2.1, 2.5, 2.8, 2.12 (a,d,f,g), 2.18, 2.19, 2.20, 3.1 (b,e,h), 3.2 (a,d), 3.3 (b,e), 3.4 (d), 3.5 (d), 3.12 (b,d), 3.13, 3.15.
	
	
\end{enumerate}

\end{document}