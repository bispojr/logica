\documentclass[12pt,a4paper,oneside]{article}

\usepackage[utf8]{inputenc}
\usepackage[portuguese]{babel}
\usepackage[T1]{fontenc}
\usepackage{amsmath}
\usepackage{amsfonts}
\usepackage{amssymb}
\usepackage{graphicx}
\usepackage{xcolor}
\usepackage{multicol}
% Definindo novas cores
\definecolor{verde}{rgb}{0.25,0.5,0.35}

\author{\\Universidade Federal de Jataí (UFJ)\\Bacharelado em Ciência da Computação \\Lógica para Ciência da Computação \\Esdras Lins Bispo Jr.}

\date{03 de julho de 2019}

\title{\sc \huge Prova (Parte 1)}

\begin{document}

\maketitle

{\bf ORIENTAÇÕES PARA A RESOLUÇÃO}

\small
 
\begin{itemize}
	\item A avaliação é individual, sem consulta;
	\item A pontuação máxima desta avaliação é 10,0 (dez) pontos, sendo uma das 06 (seis) componentes que formarão a média final da disciplina: quatro mini-testes (MT), uma prova final (PF), exercícios em formato de {\it Quizzes} (QZ) e questões conceituais (QC) aplicadas em sala de aula pelo método de Instrução pelos Colegas;
	\item A média final ($MF$) será calculada assim como se segue
	\begin{eqnarray}
		MF & = & MIN(10, S) \nonumber \\
		S & = & [(\sum_{i=1}^{4} max(MT_i, SMT_i ) + PF].0,2  + QC + QZ\nonumber
	\end{eqnarray}
	em que 
	\begin{itemize}
		\item $S$ é o somatório da pontuação de todas as avaliações, e
		\item $SMT_i$ é a substitutiva do mini-teste $i$.
	\end{itemize}
	\item O conteúdo exigido desta avaliação compreende o seguinte ponto apresentado no Plano de Ensino da disciplina: (1) Lógica Proposicional, e (2) Relações em Lógica Proposicional.
\end{itemize}

\begin{center}
	\fbox{\large Nome: \hspace{10cm}}
\end{center}

\newpage

\begin{enumerate}
	
	\section*{Primeiro Teste}
	
	\item (5,0 pt) Sejam as proposições $p$: ``O software foi testado'', e $q$: ``O usuário encontrou erros''. Traduza as duas proposições abaixo:
	\begin{enumerate}
		\item (2,5 pt) {\bf [para a linguagem natural]} \\$\sim (p \vee q)$
		\vspace*{0.3cm}
		
		{\color{blue} {\bf R:} É falso que o software foi testado ou o usuário encontrou erros.\\
		}
		
		\item (2,5 pt) {\bf [para a linguagem simbólica]} \\Se o software foi testado então o usuário não encontrou erros.
		\vspace*{0.3cm}
		
		{\color{blue} {\bf R:} $p \rightarrow \sim q$
		}
	\end{enumerate}
	
	\newpage
	
	\item (5,0 pt) Informe os todos valores lógicos das duas regiões 3x3 (R1 e R2) que estão faltando na tabela-verdade abaixo.
	
	\begin{center}
		\begin{tabular}{|c|c|c|c|c|c|c|c|c|c|}
		\hline
		$\sim$ & $p$ & $\vee$ & $q$ & $\leftrightarrow$ & $(p$ & $\rightarrow $ & $r)$ & $\wedge$ & $s$  \\
		\hline
		F & V & V & V & V & V & V & V & V & V \\
		F & V & V & V & {\color{blue} F} & {\color{blue} V} & {\color{blue} V} & V & F & F \\
		F & V & V & V & {\color{blue} F} & {\color{blue} V} & {\color{blue} F} & F & F & V \\
		F & V & V & V & {\color{blue} F} & {\color{blue} V} & {\color{blue} F} & F & F & F \\
		F & V & F & F & {\color{blue} F} & {\color{blue} V} & {\color{blue} V} & V & V & V \\
		F & V & F & F & V & V & V & V & F & F \\
		F & V & F & F & V & V & F & F & F & V \\
		F & V & F & F & V & V & F & F & F & F \\
		V & F & V & V & V & F & V & V & V & V \\
		V & F & V & V & F & F & V & V & F & F \\
		{\color{blue} V} & {\color{blue} F} & {\color{blue} V} & {\color{blue} V} & V & F & V & F & V & V \\
		{\color{blue} V} & {\color{blue} F} & {\color{blue} V} & {\color{blue} V} & F & F & V & F & F & F \\
		{\color{blue} V} & {\color{blue} F} & {\color{blue} V} & {\color{blue} F} & V & F & V & V & V & V \\
		{\color{blue} V} & {\color{blue} F} & {\color{blue} V} & {\color{blue} F} & F & F & V & V & F & F \\
		V & F & V & F & V & F & V & F & V & V \\
		V & F & V & F & F & F & V & F & F & F \\
		\hline
	\end{tabular}
\end{center}

	\newpage
	
	\section*{Segundo Teste}
	
	\item (5,0 pt) {\bf [Alencar 6.3 (f)]} Demonstrar por tabela-verdade que \\$(p \rightarrow q) \vee (p \rightarrow r) \Leftrightarrow p \rightarrow q \vee r$.
	
	\vspace*{0.3cm}
	
	{\color{blue}
		
	A proposição bicondicional associada à equivalência é $(p \rightarrow q) \vee (p \rightarrow r) \leftrightarrow p \rightarrow q \vee r$.
	
	\begin{center}
		\begin{tabular}{|c|c|c|c|c|c|c|c|c|c|c|c|c|}
			\hline
			$(p$ & $\rightarrow$ & $q)$ & $\vee$ & $(p$ & $\rightarrow$ & $r)$ & $\leftrightarrow$ & $p$ & $\rightarrow$ & $q$ & $\vee$ & $r$  \\
			\hline
			V & V & V & V & V & V & V & V & V & V & V & V & V\\
			V & V & V & V & V & F & F & V & V & V & V & V & F\\
			V & F & F & V & V & V & V & V & V & V & F & V & V\\
			V & F & F & F & V & F & F & V & V & F & F & F & F\\
			F & V & V & V & F & V & V & V & F & V & V & V & V\\
			F & V & V & V & F & V & F & V & F & V & V & V & F\\
			F & V & F & V & F & V & V & V & F & V & F & V & V\\
			F & V & F & V & F & V & F & V & F & V & F & F & F\\
			\hline
		\end{tabular}
	\end{center}
	
	Como a bicondicional é uma tautologia, então a equivalência é verdadeira $\blacksquare$
	}
	
	\item (5,0 pt) {\bf [Alencar 6.6 (b) Adaptado]} Demonstrar por tabela-verdade que \\$p \vee q  \Rightarrow (p \downarrow q) \downarrow (p \downarrow q)$.
	
	\vspace*{0.3cm}
	
	{\color{blue}
		
		A proposição condicional associada à implicação é $p \vee q  \rightarrow (p \downarrow q) \downarrow (p \downarrow q)$.
		
		\begin{center}
			\begin{tabular}{|c|c|c|c|c|c|c|c|c|c|c|}
				\hline
				$p$ & $\vee$ & $q$ & $\rightarrow$ & $(p$ & $\downarrow$ & $q)$ & $\downarrow$ & $(p$ & $\downarrow$ & $q)$  \\
				\hline
				V & V & V & V & V & F & V & V & V & F & V \\
				V & V & F & V & V & F & F & V & V & F & F \\
				F & V & V & V & F & F & V & V & F & F & V \\
				F & F & F & V & F & V & F & F & F & V & F \\
				\hline
			\end{tabular}
		\end{center}
		
		Como a condicional é uma tautologia, então a implicação é verdadeira $\blacksquare$
	}
	


\end{enumerate}

\end{document}