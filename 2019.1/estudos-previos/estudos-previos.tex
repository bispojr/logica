\documentclass[12pt,a4paper,oneside]{article}

\usepackage[utf8]{inputenc}
\usepackage[portuguese]{babel}
\usepackage[T1]{fontenc}
\usepackage{amsmath}
\usepackage{amsfonts}
\usepackage{amssymb}

\usepackage{multirow}
\usepackage{array,graphicx}

\usepackage{xcolor}
% Definindo novas cores
\definecolor{verde}{rgb}{0.25,0.5,0.35}
\definecolor{jpurple}{rgb}{0.5,0,0.35}

\author{\\Universidade Federal de Jataí (UFJ)\\Bacharelado em Ciência da Computação \\Lógica para Ciência da Computação - 2019.1 \\Prof. Esdras Lins Bispo Jr.}
\date{}

\title{
	\sc \huge Leituras para \\os Estudos Prévios
	\\{\tt Versão 1.0}
}

\begin{document}

\maketitle

\section{Livro de Referência}
	\begin{itemize}
		\item[] {\bf \color{blue} [L1]} ALENCAR FILHO, E. {\bf Iniciação à Lógica Matemática}, 1a Edição, Editora Nobel, 2002. {\color{blue} \bf Código Bib.: [510.6 ALE/ini]}.
		\item[] {\bf \color{purple} [L2]} SEBESTA, R. W.. {\bf Conceitos de Linguagem de Programação}, 9a Edição, Editora Bookman, 2011. { \color{purple} \bf Código Bib.: [004.43 SEB/con]}.
	\end{itemize}
	
\section{Trechos do Livro}

\begin{itemize}
	
	\subsection{Mini-Teste 1}
	
	\item[] {\bf Aula 02 (19/03)}: Capítulo 1 (pp. 11-15) {\bf \color{blue} [L1]}
	\item[] {\bf Aula 03 (20/03)}: Seção 2.1 a 2.4 (pp. 17-21) {\bf \color{blue} [L1]}
	\item[] {\bf Aula 04 (26/03)}: Seção 2.5 a 2.7 (pp. 21-25) {\bf \color{blue} [L1]}
	\item[] {\bf Aula 05 (27/03)}: Seção 3.1 a 3.4 (pp. 29-36) {\bf \color{blue} [L1]}
	\item[] {\bf Aula 06 (02/04)}: Seção 3.5 a 3.7 (pp. 36-39) {\bf \color{blue} [L1]}
	
	\subsection{Mini-Teste 2}
	
	\item[] {\bf Aula 09 (16/04)}: Capítulo 4 (pp. 43-48) {\bf \color{blue} [L1]}
	\item[] {\bf Aula 10 (17/04)}: Capítulo 5 (pp. 49-53) {\bf \color{blue} [L1]}
	\item[] {\bf Aula 11 (23/04)}: Seção 6.1 a 6.4 (pp. 55-59) {\bf \color{blue} [L1]}
	\item[] {\bf Aula 12 (24/04)}: Seção 6.5 a 6.7 (pp. 59-63) {\bf \color{blue} [L1]}
	
	\subsection{Mini-Teste 3}
	
	\item[] {\bf Aula 15 (08/05)}: Capítulo 7 (pp. 67-74) {\bf \color{blue} [L1]}
	\item[] {\bf Aula 16 (14/05)}: Capítulo 10 (pp. 99-110) {\bf \color{blue} [L1]}
	\item[] {\bf Aula 17 (15/05)}: Seção 9.1 a 9.5 (pp. 87-91) {\bf \color{blue} [L1]}
	\item[] {\bf Aula 18 (21/05)}: Seção 9.6 a 9.7 (pp. 91-96) {\bf \color{blue} [L1]}
	\item[] {\bf Aula 19 (22/05)}: Capítulo 11 (pp. 112-118) {\bf \color{blue} [L1]}
	
	\subsection{Mini-Teste 4}
	
	\item[] {\bf Aula 22 (04/06)}: Seção 13.1 a 13.2 (pp. 145-149) {\bf \color{blue} [L1]}
	\item[] {\bf Aula 23 (05/06)}: Seção 13.3 a 13.4 (pp. 149-153) {\bf \color{blue} [L1]}
	\item[] {\bf Aula 24 (11/06)}: Seção 16.1 a 16.5 (pp. 727-736) {\bf \color{purple} [L2]}
	\item[] {\bf Aula 25 (12/06)}: Seção 16.6.1 a 16.6.6 (pp. 736-746) 
	{\bf \color{purple} [L2]}
	\item[] {\bf Aula 26 (18/06)}: Seção 16.6.7 a 16.8 (pp. 746-749) 
	{\bf \color{purple} [L2]}
	
\end{itemize}

\end{document}