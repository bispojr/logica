\documentclass[12pt,a4paper,oneside]{article}

\usepackage[utf8]{inputenc}
\usepackage[portuguese]{babel}
\usepackage[T1]{fontenc}
\usepackage{amsmath}
\usepackage{amsfonts}
\usepackage{amssymb}
\usepackage{graphicx}
\usepackage{xcolor}
\usepackage{multicol}
% Definindo novas cores
\definecolor{verde}{rgb}{0.25,0.5,0.35}

\author{\\Universidade Federal de Jataí (UFJ)\\Bacharelado em Ciência da Computação \\Lógica para Ciência da Computação \\Esdras Lins Bispo Jr.}

\date{05 de junho de 2019}

\title{\sc \huge Mini-Teste 3}

\begin{document}

\maketitle

{\bf ORIENTAÇÕES PARA A RESOLUÇÃO}

\small
 
\begin{itemize}
	\item A avaliação é individual, sem consulta;
	\item A pontuação máxima desta avaliação é 10,0 (dez) pontos, sendo uma das 06 (seis) componentes que formarão a média final da disciplina: quatro mini-testes (MT), uma prova final (PF), exercícios em formato de {\it Quizzes} (QZ) e questões conceituais (QC) aplicadas em sala de aula pelo método de Instrução pelos Colegas;
	\item A média final ($MF$) será calculada assim como se segue
	\begin{eqnarray}
		MF & = & MIN(10, S) \nonumber \\
		S & = & [(\sum_{i=1}^{4} max(MT_i, SMT_i ) + PF].0,2  + QC + QZ\nonumber
	\end{eqnarray}
	em que 
	\begin{itemize}
		\item $S$ é o somatório da pontuação de todas as avaliações, e
		\item $SMT_i$ é a substitutiva do mini-teste $i$.
	\end{itemize}
	\item O conteúdo exigido desta avaliação compreende o seguinte ponto apresentado no Plano de Ensino da disciplina: (3) Demonstrações.
\end{itemize}

\begin{center}
	\fbox{\large Nome: \hspace{10cm}}
\end{center}

\newpage

\begin{enumerate}
	
	\section*{Terceiro Teste}
	
	\item (5,0 pt) {\bf [Alencar 9.3]} Indicar a {\bf Regra de Inferência} que justifica a {\bf validade} dos seguintes argumentos:
	\begin{enumerate}
		\item $p \rightarrow q$ \ $\vdash$ \ $(p \rightarrow q)$ $\vee \sim r$
		\item $p \rightarrow q$, $q \rightarrow \sim r$ \ $\vdash$ \ $p \rightarrow \sim r$
		\item $p \rightarrow q \vee r$ \ $\vdash$ \ $p \rightarrow p \wedge (q \vee r)$
		\item $3 <5$ \ $\vdash$ \ $3<5$ \ $\vee$ \ $3<2$
		\item $x < 0$ \ $\vee$ \ $x=1$, $x \not=1$ \ $\vdash$ \ $x <0$
	\end{enumerate}
	
	\item (5,0 pt) Verificar que são {\bf válidos} os seguintes argumentos, por meio de {\bf regras de inferência}.
	\begin{enumerate}
		\item (2,0 pt) {\bf [Alencar 11.8(c)]}
		\begin{center}
			$p \wedge q$, $p \rightarrow r$, $q \rightarrow s$ \ $\vdash$ \ $r \wedge s$
		\end{center}
		
		
		\item (3,0 pt) {\bf [Alencar 11.15(d)]} 
		\begin{center}
			$p \vee q$, $q \rightarrow r$, $p \rightarrow s$, $\sim s$ 
			\ $\vdash$ \ $r \wedge (p \vee q)$
		\end{center}
		
	\end{enumerate}

\end{enumerate}

\newpage

\section*{Regras de Inferência}

\begin{itemize}
	\item Regra da Adição (AD) \\
	(i) $p$ $\vdash$ $p\vee q$ \hspace*{0.5cm} (ii) $p$ $\vdash$ $q \vee p$
	\item Regra da Simplificação (SIMP) \\
	(i) $p \wedge q$ $\vdash$ $p$ \hspace*{0.5cm} (ii) $p \wedge q$ $\vdash$ $q$
	\item Regra da Conjunção (CONJ) \\
	(i) $p$, $q$ $\vdash$ $p \wedge q$ \hspace*{0.5cm} (ii) $p$, $q$ $\vdash$ $q \wedge p$
	\item Regra da Absorção (ABS) \\
	$p \rightarrow q$ $\vdash$ $p \rightarrow (p \wedge q)$
	\item Regra {\it Modus Ponens} (MP) \\
	$p \rightarrow q$, $p$ $\vdash$ $q$
	\item Regra {\it Modus Tollens} (MT) \\
	$p \rightarrow q$, $\sim q$ $\vdash$ $\sim p$
	\item Regra do Silogismo Disjuntivo (SD) \\
	(i) $p \vee q$, $\sim p$ $\vdash$ $q$ \hspace*{0.5cm} (ii) $p \vee q$, $\sim q$ $\vdash$ $p$
	\item Regra do Silogismo Hipotético (SH) \\
		$p \rightarrow q$, $q \rightarrow r$ $\vdash$ $p \rightarrow r$
	\item Regra do Dilema Construtivo (DC) \\
		$p \rightarrow q$, $r \rightarrow s$, $p \vee r$ $\vdash$ $q \vee s$
	\item Regra do Dilema Destrutivo (DD) \\
		$p \rightarrow q$, $r \rightarrow s$, $\sim q$ $\vee \sim s$ $\vdash$ $\sim p$ $\vee \sim r$
\end{itemize}

%\begin{block}{Regra da Conjunção (CONJ)}
%\begin{columns}
%	\column{.5\textwidth} 
%	\begin{itemize}
%		\item $p$
%		\item $q$ \\
%		\rule{3cm}{0.5pt} 
%		\item $p \wedge q$
%	\end{itemize}
%	\column{.5\textwidth} 
%	\begin{itemize}
%		\item $p$
%		\item $q$ \\
%		\rule{3cm}{0.5pt} 
%		\item $q \wedge p$
%	\end{itemize}
%\end{columns}
%\end{block}
%\end{frame}
%
%\begin{frame}{Regras de Inferência}
%\begin{block}{Regra da Absorção (ABS)}
%\begin{columns}
%\column{.5\textwidth} 
%\begin{itemize}
%	\item $p \rightarrow q$ \\
%	\rule{3cm}{0.5pt} 
%	\item $p \rightarrow (p \wedge q)$
%\end{itemize}
%\column{.5\textwidth} 
%\end{columns}
%\end{block} \pause
%\begin{block}{Regra {\it Modus Ponens} (MP)}
%\begin{columns}
%\column{.5\textwidth} 
%\begin{itemize}
%	\item $p \rightarrow q$
%	\item $p$ \\
%	\rule{3cm}{0.5pt} 
%	\item $q$
%\end{itemize}
%\column{.5\textwidth} 
%\end{columns}
%\end{block}
%\end{frame}
%
%\begin{frame}{Questão 038}
%\begin{block}{[Q038]}
%Dado o seguinte argumento válido
%\begin{center}
%$p \rightarrow (q \rightarrow r)$, $p$ $\vdash$ $q \rightarrow r$
%\end{center}
%
%Qual das regras de inferência abaixo melhor justifica a sua validade?
%\end{block}
%\begin{enumerate}[label=(\Alph*)]
%\item Adição (AD)
%\item Conjunção (CONJ)
%\item Absorção (ABS)
%\item {\it Modus Ponens} (MP)
%\end{enumerate}
%\end{frame}
%
%\begin{frame}{Regras de Inferência}
%\begin{block}{Regra {\it Modus Tollens} (MT)}
%\begin{columns}
%\column{.5\textwidth} 
%\begin{itemize}
%\item $p \rightarrow q$
%\item $\sim q$ \\
%\rule{3cm}{0.5pt} 
%\item $\sim p$
%\end{itemize}
%\column{.5\textwidth} 
%\end{columns}
%\end{block} \pause
%\begin{block}{Regra do Silogismo Disjuntivo (SD)}
%\begin{columns}
%\column{.5\textwidth} 
%\begin{itemize}
%\item $p \vee q$
%\item $\sim p$ \\
%\rule{3cm}{0.5pt} 
%\item $q$
%\end{itemize}
%\column{.5\textwidth} 
%\begin{itemize}
%\item $p \vee q$
%\item $\sim q$ \\
%\rule{3cm}{0.5pt} 
%\item $p$
%\end{itemize}
%\end{columns}
%\end{block}
%\end{frame}
%
%\begin{frame}{Questão 039}
%\begin{block}{[Q039]}
%Dado o seguinte argumento válido
%\begin{center}
%$(p \wedge q) \vee (\sim p \wedge r)$, $\sim(\sim p \wedge r)$ $\vdash$ $p \wedge q$
%\end{center}
%
%Qual das regras de inferência abaixo melhor justifica a sua validade?
%\end{block}
%\begin{enumerate}[label=(\Alph*)]
%\item {\it Modus Tollens} (MT)
%\item Silogismo Disjuntivo (SD)
%\item Simplificação (SIMP)
%\item {\it Modus Ponens} (MP)	
%\end{enumerate}
%\end{frame}
%
%\begin{frame}{Regras de Inferência}
%\begin{block}{Regra do Silogismo Hipotético (SH)}
%\begin{columns}
%\column{.5\textwidth} 
%\begin{itemize}
%\item $p \rightarrow q$
%\item $q \rightarrow r$ \\
%\rule{3cm}{0.5pt} 
%\item $p \rightarrow r$
%\end{itemize}
%\column{.5\textwidth} 
%\end{columns}
%\end{block} \pause
%\begin{block}{Regra do Dilema Construtivo (DC)}
%\begin{columns}
%\column{.5\textwidth} 
%\begin{itemize}
%\item $p \rightarrow q$
%\item $r \rightarrow s$
%\item $p \vee r$\\
%\rule{3cm}{0.5pt} 
%\item $q \vee s$
%\end{itemize}
%\column{.5\textwidth} 
%\end{columns}
%\end{block}
%\end{frame}
%
%\begin{frame}{Regras de Inferência}
%\begin{block}{Regra do Dilema Destrutivo (DD)}
%\begin{columns}
%\column{.5\textwidth} 
%\begin{itemize}
%\item $p \rightarrow q$
%\item $r \rightarrow s$
%\item $\sim q \vee \sim s$\\
%\rule{3cm}{0.5pt} 
%\item $\sim p \vee \sim r$
%\end{itemize}
%\column{.5\textwidth} 
%\end{columns}
%\end{block}
%\end{frame}

\end{document}