\documentclass[12pt,a4paper,oneside]{article}

\usepackage[utf8]{inputenc}
\usepackage[portuguese]{babel}
\usepackage[T1]{fontenc}
\usepackage{amsmath}
\usepackage{amsfonts}
\usepackage{amssymb}
\usepackage{graphicx}
\usepackage{xcolor}
\usepackage{multicol}
% Definindo novas cores
\definecolor{verde}{rgb}{0.25,0.5,0.35}

\author{\\Universidade Federal de Jataí (UFJ)\\Bacharelado em Ciência da Computação \\Lógica para Ciência da Computação \\Esdras Lins Bispo Jr.}

\date{14 de maio de 2019}

\title{\sc \huge Mini-Teste 2}

\begin{document}

\maketitle

{\bf ORIENTAÇÕES PARA A RESOLUÇÃO}

\small
 
\begin{itemize}
	\item A avaliação é individual, sem consulta;
	\item A pontuação máxima desta avaliação é 10,0 (dez) pontos, sendo uma das 06 (seis) componentes que formarão a média final da disciplina: quatro mini-testes (MT), uma prova final (PF), exercícios em formato de {\it Quizzes} (QZ) e questões conceituais (QC) aplicadas em sala de aula pelo método de Instrução pelos Colegas;
	\item A média final ($MF$) será calculada assim como se segue
	\begin{eqnarray}
		MF & = & MIN(10, S) \nonumber \\
		S & = & [(\sum_{i=1}^{4} max(MT_i, SMT_i ) + PF].0,2  + QC + QZ\nonumber
	\end{eqnarray}
	em que 
	\begin{itemize}
		\item $S$ é o somatório da pontuação de todas as avaliações, e
		\item $SMT_i$ é a substitutiva do mini-teste $i$.
	\end{itemize}
	\item O conteúdo exigido desta avaliação compreende o seguinte ponto apresentado no Plano de Ensino da disciplina: (2) Relações em Lógica Proposicional.
\end{itemize}

\begin{center}
	\fbox{\large Nome: \hspace{10cm}}
\end{center}

\newpage

\begin{enumerate}
	
	\section*{Segundo Teste}
	
	\item (5,0 pt) {\bf [IpC Q030]}	Dada uma proposição $p$ qualquer, \\qual das declarações abaixo é {\bf falsa}? {\bf Justifique a sua resposta!}
	\begin{enumerate}
		\item A contrapositiva da contrapositiva de $p$ é $p$.
		\item A contrária da contrária de $p$ é $p$.
		\item Se $V(p) = F$, então \\o valor lógico da recíproca de $p$ é verdadeiro.
		\item Se $V(p) = F$, então \\o valor lógico da contrapositiva de $p$ é verdadeiro.
	\end{enumerate}
	
	
	\item (5,0 pt) {\bf [Alencar 6.3 (e)]} Demonstrar por tabela-verdade que \\$(p \rightarrow q) \wedge (p \rightarrow r) \Leftrightarrow p \rightarrow q \wedge r$.

\end{enumerate}

\end{document}