 \documentclass[xcolor=dvipsnames,table]{beamer}
%o
%e

\usepackage{latexsym}
\usepackage [ansinew]{inputenc}
\usepackage[brazil]{babel}
\usepackage{amssymb} %Este e o AMS paquete
\usepackage{amsmath}
\usepackage{stmaryrd}
\usepackage{fancybox}
\usepackage{datetime}
\usepackage{enumitem}

\usepackage[T1]{fontenc}

%\usepackage{beamerthemesplit}
\usepackage{graphicx}
\usepackage{graphics}
\usepackage{url}
\usepackage{algorithmic}
\usepackage{algorithm}
\usepackage{acronym}
\usepackage{array}

\newtheorem{definicao}{Definio}
\newcommand{\tab}{\hspace*{2em}}

\mode<presentation>
{
  %\definecolor{colortexto}{RGB}{153,100,0}
  \definecolor{colortexto}{RGB}{0,0,0}
  
% \setbeamersize{sidebar width left=0.5cm}
  \setbeamertemplate{background canvas}[vertical shading][ bottom=white!10,top=white!10]
%   \setbeamercolor{title}{fg=colortitulo!80!black,bg=blue!20!white}
%   \setbeamercolor{title}{bg=colortitulo!20!black}
%   \setbeamercolor{background canvas}{bg=colortitulo}
%   \setbeamercolor{frametitle}{fg=red}

  \setbeamercolor{normal text}{fg=colortexto} 

  \usetheme{Warsaw}
  %\logo{\includegraphics[width=2cm]{Images/ratonfuerte.jpg}}


%   \usefonttheme[onlysmall]{structurebold}
%   \usecolortheme{seahorse}
%  \usecolortheme[named={YellowOrange}]{structure}
%   \usecolortheme[named={Blue}]{structure}
%   \usecolortheme{crane}
%   \useoutertheme{default}
}
\setbeamertemplate{caption}[numbered]

\title{Quantificadores e Predicados} 

\author{
  Esdras Lins Bispo Jr. \\ \url{bispojr@ufg.br}
  } 
 \institute{
  L�gica para Ci�ncia da Computa��o \\Bacharelado em Ci�ncia da Computa��o}
\date{\textbf{04 de julho de 2018} }

\logo{\includegraphics[width=1cm]{images/ufgJataiLogo.png}}

\begin{document}

	\begin{frame}
		\titlepage
	\end{frame}

	\begin{frame}{Plano de Aula}
		\tableofcontents
		%\tableofcontents[hideallsubsections]
	\end{frame}
	
	\section{Instru��o pelos Colegas}

	\begin{frame}{Quest�o 050}	
		\begin{block}{[Gersting]}
			Admita $P(x,y)$ como o predicado $x \leq y$; e \\$Q(x,y)$ como $x$ divide $y$. \\Admita tamb�m o dom�nio de interpreta��o como sendo todos os n�meros inteiros positivos. Sobre a express�o
			\begin{center}
				$(\forall x) (\exists y)\left[P(x,y) \wedge Q(x,y)\right]$
			\end{center}
		a �nica alternativa {\bf incorreta} � 
		\end{block}
		\begin{enumerate}[label=(\Alph*)]
			\item o escopo de $(\forall x)$ � $(\exists y)\left[P(x,y) \wedge Q(x,y)\right]$.
			\item o escopo de $(\exists y)$ � $P(x,y) \wedge Q(x,y)$. 
			\item ela � equivalente � express�o $(\forall x) \left[(\exists y)P(x,y) \wedge Q(x,y)\right]$.
			\item seu valor-verdade � verdadeiro.
		\end{enumerate}
	\end{frame}

	\begin{frame}
		\titlepage
	\end{frame}
	
\end{document}