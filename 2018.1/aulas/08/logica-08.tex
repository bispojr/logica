 \documentclass[xcolor=dvipsnames,table]{beamer}
%o
%e

\usepackage{latexsym}
\usepackage [ansinew]{inputenc}
\usepackage[brazil]{babel}
\usepackage{amssymb} %Este e o AMS paquete
\usepackage{amsmath}
\usepackage{stmaryrd}
\usepackage{fancybox}
\usepackage{datetime}
\usepackage{enumitem}

\usepackage[T1]{fontenc}

%\usepackage{beamerthemesplit}
\usepackage{graphicx}
\usepackage{graphics}
\usepackage{url}
\usepackage{algorithmic}
\usepackage{algorithm}
\usepackage{acronym}
\usepackage{array}

\newtheorem{definicao}{Definio}
\newcommand{\tab}{\hspace*{2em}}

\mode<presentation>
{
  %\definecolor{colortexto}{RGB}{153,100,0}
  \definecolor{colortexto}{RGB}{0,0,0}
  
% \setbeamersize{sidebar width left=0.5cm}
  \setbeamertemplate{background canvas}[vertical shading][ bottom=white!10,top=white!10]
%   \setbeamercolor{title}{fg=colortitulo!80!black,bg=blue!20!white}
%   \setbeamercolor{title}{bg=colortitulo!20!black}
%   \setbeamercolor{background canvas}{bg=colortitulo}
%   \setbeamercolor{frametitle}{fg=red}

  \setbeamercolor{normal text}{fg=colortexto} 

  \usetheme{Warsaw}
  %\logo{\includegraphics[width=2cm]{Images/ratonfuerte.jpg}}


%   \usefonttheme[onlysmall]{structurebold}
%   \usecolortheme{seahorse}
%  \usecolortheme[named={YellowOrange}]{structure}
%   \usecolortheme[named={Blue}]{structure}
%   \usecolortheme{crane}
%   \useoutertheme{default}
}
\setbeamertemplate{caption}[numbered]

\title{Introdu��o � L�gica Proposicional} 

\author{
  Esdras Lins Bispo Jr. \\ \url{bispojr@ufg.br}
  } 
 \institute{
  L�gica para Ci�ncia da Computa��o \\Bacharelado em Ci�ncia da Computa��o}
\date{\textbf{02 de maio de 2018} }

\logo{\includegraphics[width=1cm]{images/ufgJataiLogo.png}}

\begin{document}

	\begin{frame}
		\titlepage
	\end{frame}

	\begin{frame}{Plano de Aula}
		\tableofcontents
		%\tableofcontents[hideallsubsections]
	\end{frame}
	
	\section{Instru��o pelos Colegas}
	
	\begin{frame}{Quest�o 031}	
		\begin{block}{[Gersting]}
			Sejam $P$ a express�o ``$x > 5$'' e $Q$ a express�o ``$y < 3$''. \\Em Pascal, a express�o $P \vee Q'$ � dada por...
		\end{block}
		\begin{enumerate}[label=(\Alph*)]
			\item {\bf not}($x > 5$ {\bf or} $y < 3$)
			\item $x > 5$ {\bf or} $y < 3$ {\bf not}
			\item {\color{blue} $x > 5$ {\bf or} {\bf not} $y < 3$}
			\item {\bf not} $x > 5$ {\bf or} $y < 3$
		\end{enumerate}
	\end{frame}

	\begin{frame}{Quest�o 032}	
		\begin{block}{[Gersting]}
			A express�o em Pascal 
			\begin{center}
				$x > 5$ {\bf and not} $x > 5$ {\bf or} $y < 3$
			\end{center}
			pode ser simplificada para 
		\end{block}
		\begin{enumerate}[label=(\Alph*)]
			\item $x > 5$
			\item {\color{blue} $y < 3$}
			\item {\it false}
			\item n�o pode ser simplificada
		\end{enumerate}
	\end{frame}

	\begin{frame}{Quest�o 033}	
		\begin{block}{[Gersting]}
			A proposi��o $(A \rightarrow B) \rightarrow (B' \rightarrow A')$ � uma tautologia, pois
		\end{block}
		\begin{enumerate}[label=(\Alph*)]
			\item $A \rightarrow B$ sempre � verdadeira.
			\item $B' \rightarrow A'$ nunca � falsa.
			\item � imposs�vel um condicional assumir valor-verdade falso.
			\item {\color{blue} � imposs�vel $A \rightarrow B$ ser verdadeira e $B' \rightarrow A'$ ser falsa.}
		\end{enumerate}
	\end{frame}

	\begin{frame}{Quest�o 034}	
		\begin{block}{[Gersting]}
			A proposi��o $(A \rightarrow (B \rightarrow C)) \rightarrow (A \wedge B \rightarrow C)$ � uma tautologia?
		\end{block}
		\begin{enumerate}[label=(\Alph*)]
			\item {\color{blue} Sim}
			\item N�o
			\item Depende apenas do valor de $B$.
			\item n�o � poss�vel definir.
	\end{enumerate}
	\end{frame}

	\begin{frame}
		\titlepage
	\end{frame}
	
\end{document}